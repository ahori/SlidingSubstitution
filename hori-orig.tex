\documentclass[10pt,conference,a4paper,fleqn]{IEEEtran}

\usepackage[cmex10]{amsmath}
\usepackage{times,epsfig}
\usepackage{pseudocode}

\long\def\comment#1{}

\title{A Study on Communication Performance Degradation Using Spare Nodes}

\author{
\IEEEauthorblockN{Atsushi Hori, Kazumi Yoshinaga, Yutaka Ishikawa}
\IEEEauthorblockA{RIKEN Advanced Institute for Computational Science\\
7-1-26 Minatojima-minami-machi, Chuo-ku, Kobe, Hyogo 650-0047, JAPAN\\
ahori@riken.jp, kazumi.yoshinaga@riken.jp, yutaka.ishikawa@riken.jp}
}

\begin{document}
\maketitle

\begin{abstract}

Fault tolerance is gathering attentions since the fault rate of
high-end supercomputers is increasing. To cope with this situation,
the fault tolerance techniques have been being investigated.
User-level algorithm-based fault-tolerance techniques and the
parallel execution environments enabling jobs to continue their
execution after node failure happens are so far proposed and
developed. Even using these, some programs, such as stencil
computation, having static load balancing are very hard to continue
their execution efficiently. In this case, having some spare nodes and
substituting failed with the spare nodes seem to be an effective way.

This paper focuses on this topic, how spare node should be allocated,
how fault nodes must be substituted by spare nodes, and how much
communication performance affected by the substitution. The third
question come from that node-rank mapping changed by the substitution
and thus there is a possibility of having extra message collisions. In
a stencil computation, node-rank mapping is done in a straight forward
way on a Cartesian network without incurring any message
collisions. However, once the substitution happens, this 
node-rank mapping might be gone. Therefore, above three questions must
be answered to minimize the communication performance degradation.

In this paper, several spare node allocation and node substitution
methods will be proposed, analyzed and compared in terms of
communication performance after the substitution. It will be shown
that the P2P communication performance on the K computer can be 1.6
times slower than that without any failure. The alltoall collective
communication on the K computer can take more than twice when the
substitution happens. However, those numbers can be reduced by using
a proper substitution method.

\end{abstract}

\begin{IEEEkeywords}
fault tolerance; fault mitigation; spare node; communication performance;
\end{IEEEkeywords}

\section{Introduction}

Fault tolerance is gathering attentions since the fault rate of
high-end supercomputers is increasing\cite{techrep10040} and
now users are experiencing that their jobs are aborted due to some
system errors\cite{6903615}. To cope with this situation, the fault
tolerance techniques have been being investigated. Checkpoint and
restart for parallel jobs is a well-known technique which enables jobs
can continue their execution from a previous checkpoint (there are so
many papers on checkpoint and restart, however, the most notable one
is \cite{CLIP}).  

Soon after applying the checkpoint technique to the jobs running on
large parallel machines, it was recognized that the I/O to save the 
checkpoint data became a bottleneck. Much research effort on this
topic is being propelled (\cite{Sato:2012:DMN:2388996.2389022}, for
example). On the other hand, user-level checkpoint where user
program takes care of checkpoint and restart attracts
researchers and programmers. Since user program knows  
which data to be saved and which data not to be saved, the amount of
checkpoint data can be drastically reduced, and thus, the I/O time can
be reduced so much with the sacrifice of extra user programing. 

Davies {\it et al.} showed that a user program can be fault-tolerant
without checkpointing\cite{Davies:2011:HPL:1995896.1995923}. In this
technique, the parity to recover the lost data can be embedded into the
LU decomposition algorithm and user program can survive from failure
without checkpointing. This is a great discovery, however, it
opens a new research topic. Users can write a program which can handle
fault, however, lower-level system software does not allow users to do
it. MPI is a most widely-used communication library and its
specification is well-defined\cite{mpi-v3}. Unfortunately, the current
MPI standard defines nothing when a fault happens in the middle of MPI 
communication. Thus, the user-level fault handling cannot be
implemented with the current MPI standard.

To define the MPI behaviour in the presence of fault, User-level
Failure Mitigation (ULFM) is proposed and being developed to handle
process failures and node failures\cite{Bland01082013}. ULFM provides
the API so that the modification of the existing MPI specification is 
minimized. Even using ULFM, user-level fault handling is not easy at
all, and some frameworks are proposed to alleviate user effort.
Falanx is a fault tolerant framework for master-worker programing
model. Local Failure Local Recovery (LFLR) is another fault tolerant
framework\cite{Teranishi:2014:TLF:2642769.2642774} which covers wider 
programing model than that Falanx can support. Both Falanx and LFLR
are implemented by using ULFM. Global View Resilience (GVR) is another
user-level fault mitigation system based on PGAS programing
model\cite{gvr-sc14,GVR-vecpar2014}.

We believe that the user-level fault handling code must be as simple
as possible. The situation where the code of handling the first
node failure is different from the code of handling of second or more
number of node failure must be avoided, because it is very hard to
produce these situation to test the program. This kind of complexity
must be hidden by a system software.

\begin{figure}[ht]
\begin{center}
\psfig{figure=Figs/NodeFailure.eps,width=80mm}
  \caption{An example of node failure and recovery} 
  \label{fig:node-failure}
\end{center}
\end{figure}

Figure\ref{fig:node-failure} shows an example of a node failure in a
2D network consisting of 36 nodes. Here, it is assumed that a job is
running on this machine, and the job is written with a fail-stop-free
runtime system, such as ULFM. When the node 21 goes down (left of
Figure\ref{fig:node-failure}, The job running on those 36 nodes can
take one of the following actions; 

\begin{itemize}
\item Abort the job execution and resubmit the job (from a previous
  checkpoint, if possible), or
\item The remaining 35 nodes take over the job.
\end{itemize}

In the former strategy, the user-level fault handling can do nothing. 
In the latter strategy, the task allocated to the failed node must
be leveled by the remaining 35 nodes, otherwise load imbalanced incurs
extra overhead in addition to the loss of one node. If the job has the
capability of dynamic load balancing, such as master-worker models or
particle-in-cell simulations, then the load imbalance can be leveled by
the application itself without modifying the code so much. However, if
the job is a stencil application, this kind of applications does not
have dynamic load balancing capability in most cases, fault handling
is not easy at all. In most stencil computation, communication pattern
is static and load balancing is also static. To preserve the
communication pattern, one possible idea to handle a node failure is
to exclude the row and column including the failed node (middle of
Figure\ref{fig:node-failure}) to preserve the stencil communication
pattern. However, the task allocated to the 
failed node must be leveled with the other running nodes (right of
Figure\ref{fig:node-failure}). This load-leveling is an additional
code to handle the fault, and this must be avoided if possible.
\comment{ This is
  to say that there is a job class   which can handle node failure
  easily and the job class which cannot handle node 
  failure easily. }

If a system software reserves a set of spare nodes in advance, and the
failed node is replaced by the spare node, then the user-level  
node failure handling can be much easier, because the number of nodes
involved in the computation is the same. LFLR assumes the use of
spare node. GVR, although detailed recovery process is hidden from
users, may utilize spare nodes. However, at the best of our knowledge,
there is no discussion how to reserve spare nodes and how to replace
a failed node with a spare node. As an evaluation index, we chose
communication performance. This is because the use of spare nodes may
introduce extra message collisions. We will explain this situation in
the next section.

This paper reports our investigation result on these issues. We
propose several spare node replacement methods. The proposed methods
are discussed and compared from the viewpoint of communication
performance degradation. The most contribution of this paper is
proposing and comparing spare node allocation and substitution for
fault nodes by focusing on communication performance degradation.

\section{Using Spare Nodes}

Hereafter, the network has a multi-dimensional Cartesian (mesh and/or
torus) topology. This is because 4 machines out of the top 5 machines
listed in the TOP500\cite{top500} as of November 2014 have a network
with such a topology\ref{tbl:top5-network}. 

\begin{table}[htb]
\centering
\caption{Network topology of Top5 machines\cite{top500}}
\label{tbl:top5-network}
\begin{tabular}{c|l|r|l} 
\hline
Top500 rank & Name & \# Cores & Topology \\
\hline
1 & Tianhe-2 & 3,120,000 & FatTree \\
2 & Titan (Cray XK7) & 560,640 & 3D Torus \\
3 & Sequoia (BG/Q) & 1,572,864 & 5D Torus/Mesh \\
4 & the K computer & 705,024 & 6D Torus/Mesh \\ 
5 & Mira (BG/Q) & 786,432 & 5D Torus/Mesh \\
\hline
\end{tabular}
\end{table}

From the programmers point of view, having spare nodes and fault nodes
substituted by the spare nodes is not so difficult. With MPI, the
modification is 1) a new MPI communicator is created where the fault
node is extracted (on ULFM, {\tt MPI\_Comm\_shrink} do this) and a
selected spare node is added in, 2) the spare node is set up to take
over the execution on the failed node. The other part of the program
can be as it was. This means that the logical topology provided by
the new MPI communicator can be the same as before the failure,
however, the actual physical topology cannot be the same. New message
collisions which cannot be seen when the system has no fault node can
be seen when some nodes go down. 

Therefore, fault nodes substitution with spare nodes must be
carefully done to minimize the communication performance
degradation. There are many aspects to be considerd in this, not only
the communication performance, but also system utilization, job turn
around time, easiness of user programing, framework to be developed,
and so on. Unfortunately, there is almost no research has been done so
far. So, in this paper, we pay attention to the communication
performance mostly. 

Throughout this paper, only the node failure is our concern. Network
failure may happen but it is assumed that the network failure can be
recovered by the network itself\cite{Domke:2014:FND:2683593.2683659}
(see also Section\ref{sec:related-work}). The Tofu network which is
used by the K computer has some redundant links to detour the failed
node\cite{sumimoto-k}. It is assumed that a job can survive even with
the node failure(s) by using a parallel computing environment which
provides user-level fault mitigation mechanism such as ULFM, and the
process(es) running on a failed node can be recovered anyhow, from a
checkpoint or by using parity of the other alive processes. Finally,
It is also assumed that the processes running on a node can be
migrated to any other node.  

In the next subsection, spare node allocation will be discussed. And 
the possibility of degradation of communication performance will be
shown. Then, three methods of substitution of spare node for fault
node will be proposed and compared. 

\subsection{Spare Node Allocation}\label{sec:spare-alloc}

Just for simplicity, 2D network is used hereinafter for
explanation. On the higher dimensional networks with the dimension
order routing (the higher order of XY
routing\cite{Zhang:2009:CRX:1603897.1605067}) where the messages are 
routed in the dimension order, and this results in that the messages
are eventually routed to the plane consisting of the last two
dimensions. Thus, the discussion using the 2D network is meaningful. 

Figure\ref{fig:sparenode-allocation} shows three different spare node
allocations. Each small square represents a node. Left one in this figure,
a column of right side of the 2D plain is reserved for spare node
set. In the middle one, 2 sides (right and bottom) are reserved for
spare node set. In the rightmost one, 2 sides having 2 nodes thickness
are reserved. Those spare node reservations are denoted as {\it
  2D(1,1)}, {\it 2D(2,1)} and {\it 2D(2,2)}, respectively. ``2D''
means that the reservation is done on the 2D plane. The first number
in the bracket is the number of sides where spare nodes reside. The
second number is the thickness, number of spare nodes on each side. 

\begin{figure}[ht]
\begin{center}
\psfig{figure=Figs/SpareNodeAllocation.eps,width=80mm}
  \caption{Spare node allocation} 
  \label{fig:sparenode-allocation}
\end{center}
\end{figure}

Spare nodes are allocated at the side(s) of 2D plain in Figure
\ref{fig:sparenode-allocation} so that a stencil computation
having non-periodical boundaries will not have any overhead. 
This is not true with the stencil computation having periodic
boundaries and the network having torus topology. However, the
overhead comes from the increased hop count by one. This is very small
(100ns per hos, in the K computer). 

The percentage of the number of spare nodes to the number of nodes
allocated for this job in the 2D(2,2) case can be expressed in the
following expression. 

\[
R_{2D(2,2)} = 1 - ( N^{1/2} - 2 )^2 / N
\]

In more general, $qD(r,s)$ case, the spare node percentage can be
expressed in the following expression.

\[
R_{qD(r,s)} = 1 - \frac{ ( N^{1/q} - s )^r \times ( N^{1/q} )^{q-r} }{ N }
\]

Here, $r \leq q$ and $s < N^q$. Note that this expression is not
precise because the number of nodes are integer and ignoring the
flooring effect. However, this graph can be useful for how spare node
percentage will be over the number of nodes for a job. 

Figure \ref{fig:sparenode-percentage} shows the percentage of spare
nodes based on the expression. As shown in this
figure, the higher the order of network dimensions, the higher the
percentage of spare nodes. The percentage is almost proportional to
the number of sides used for the spare nodes. Most notably, the larger
the job size, the lower the percentage. We will discuss on this point
later in Section \ref{label:multi-job}.

\begin{figure}[ht]
\begin{center}
\psfig{figure=Figs/R-SpareNodes.eps,width=60mm}
  \caption{Percentage of spare nodes} 
  \label{fig:sparenode-percentage}
\end{center}
\end{figure}

There can be the case, 4 sides in a 2D plane can be allocated for the
spare nodes. On a torus network, however, this is almost equal to the
2D(2,2) case. In the following discussion, we could not find any
significant difference between 2D(4,1) and 2D(2,2) for example, and
thus, we omit the cases where $r > q$. The thickness, $s$, does not
affect the nature of the spare node  substitution method described in
Section\ref{sec:substitution}, so we investigated the cases having the
thickness of one. 

Having a spare nodes set can decrease the system utilization
ratio. However, this does not happen always. On the K computer, the
size of each dimension of a job must be in a {\it Tofu unit} which has
12 nodes. When a user
submits a job to have the 3D shape of 11x11x11, for
example, the job may have 12x12x12 nodes when the job is
scheduled. This results in having 3D(3,1) spare nodes. The same
situations can be seen on the other big 
machines having Cartesian topology network listed in Table
\ref{tbl:top5-network}. On BG/Q machines, the number of nodes of a job
must be the number of power of 2\cite{BGQ}. On Cray XK/7, a job is
allocated with using the 4 \time 2 \time 8
blocks\cite{Pena:2013:ATM:2488551.2488564}. Thus, the gap between the
number of nodes required by a job and the number of nodes actually
allocated can be used for the spare node set without having any extra
cost. 

\subsection{Substitution of Spare Node for Fault Node}
\label{sec:substitution}

Communication performance degradation can be seen because the spare
node by which fault node is substituted can be allocated far from 
the original node. Figure \ref{fig:collisions} shows the 5P-stencil
communication pattern (left). In the 5P-stencil communication on a
Cartesian topology, no message collision happens because a node
communicates with its neighbor nodes only. Here, the XY(Z)
routing is assumed. On the right
of Figure \ref{fig:collisions}, however, when a fault node (denoted as
``F'') is substituted by a spare node (denoted as ``S''), the beauty
of the stencil communication pattern is lost. As shown in this 
figure, there are 5 message route at the links in the dotted
ellipse. This means up to 5 message collisions may happen.

\begin{figure}[ht]
\begin{center}
\psfig{figure=Figs/Collision.eps,width=80mm}
  \caption{Message Collisions} 
  \label{fig:collisions}
\end{center}
\end{figure}

Figure \ref{fig:node-substitution} shows three node substitution
methods we proposed. We name them {\it 0D}, {\it 1D} and {\it 2D}
sliding methods. With the higher dimension networks, 3D or more, the
those proposed methods can be augmented in a natural way. Just for
simplicity, we will explain them on 2D network topology only. And we use
the 5P-stencil communication pattern in many cases, where 4 massages, up,
down, left, right are sent from a node. In the 9P-stencil
communication pattern, extra 4 messages are sent to diagonal
directions. However, in most cases, the length of those diagonal
messages is much shorter than the messages in the 5P-stencil
pattern and the effect on the communication performance is considered
to be small.

\subsubsection{0D Sliding}\label{sec:0d-sliding}

The 0D sliding method is the simplest one among the others. The fault node
is simply substituted with a spare node (as already shown in Figure
\ref{fig:node-substitution}. There is a big drawback in this
method. When a node failure happens on the far end from the spare node
set, then the hop distance from the failed node and the spare node can
be very large. This means the possibility of message collisions due to
the replacement and the higher communication latency due to the large
number of hop counts. To minimize this, the spare node having the
shortest Manhattan distance from the failed node should be chosen. 

\begin{figure}[ht]
\begin{center}
\psfig{figure=Figs/SpareNode.eps,width=80mm}
  \caption{Fault node substitution methods} 
  \label{fig:node-substitution}
\end{center}
\end{figure}

Figure \ref{fig:0d-worst-cases} shows examples of the 0D sliding results
with 2D(1,1) spare node allocation when multiple faults happened. In
the left, Nodes, 1, 2, 4, and 5, are down and substituted spare nodes are
1', 2', 3', 4', 5', respectively. Those spare nodes are chosen so that
the hop counts between the fault node and corresponding spare node
minimized. In the non-periodic 5P-stencil communication in the XY
routing algorithm, 
the messages from all of the spare nodes to the adjacent nodes A, B,
C, D, E, F of the failed nodes are firstly centralized to the node 1'
(because the X direction routing of the XY routing algorithm) and then
go to the target nodes. Thus, the network links between 1' and A has
ten messages plus one which is the normal stencil communication message
between the nodes (shown in white boxes between 1' and A), and this is
11. This is the worst case in 0D sliding and the number of fault nodes
is less than or equal to 6. 

\begin{figure}[ht]
\begin{center}
\psfig{figure=Figs/0D-WorstCases.eps,width=80mm}
  \caption{Worst cases in 0D sliding} 
  \label{fig:0d-worst-cases}
\end{center}
\end{figure}

The right of Figure \ref{fig:0d-worst-cases}  shows the case where the
network topology is 2D mesh, spare nodes are reserved in 2D(1,1),
faults happen in a row or column, and the row or column is close to
the side of the network. The failed node 1 is replaced by spare node
1' and so on. In this case, the failures 
happen close to the side of the network and the way the spare node
selection shown in the left of Figure \ref{fig:0d-worst-cases} is
impossible. In the non-periodic 5P-stencil communication, all messages
from the spare nodes 4', 5', 6' and 7' to the logically neighbor nodes
A to V go through the link between 3' and 4'. The number of messages
is 16, since each node sends 4 messages to its neighbor 4 nodes. This
situation can happen when the number of faults is larger than or equal
to 7. The next expression is the relation between the maximum number
of possible message collisions ($C_{max}$) and the number of node
failure ($F_n$). Note that when the number of $C_{max}$ is one then
there is up to one message in any network links and this means no
collision happens.

\begin{equation}
C_{max} = \left \{ \begin{array}{ll}
2 \times F_n + 1 & F_n \leq 6~or~torus~topology \\
4 \times ( F_n - 3 ) & F_n \geq 7~and~mesh~topology
\end{array}\nonumber
\right.
\end{equation}

This worst case scenario can be relaxed by having spare nodes in
the 2D(2,1) way. If the failures happen in the same row or column,
then the spare node must be chosen from the different side
alternatively, for example. See also Section \ref{sec:comparison}.

Figure \ref{fig:0d-K} shows the observed communication performance
degradation measured over the number of node measured on the K
computer. Here, our benchmark program records the worst latency among
the any possible combination of a fault node and a spare node. Here,
message size is 4MiB, 5P-stencil communication pattern is sued.

\begin{figure}[ht]
\begin{center}
\psfig{figure=Figs/Scalability-K.eps,width=70mm}
  \caption{Worst 5P-stencil communication time with the 0D sliding method
    on the K computer} 
  \label{fig:0d-K}
\end{center}
\end{figure}

As shown is this graph, the latency is almost three times worse than
the latency without having any fault node. As described the maximum
possible message collisions in this case is 5. However, the Tofu
network hardware of the K computer allows to send 4 messages with 
different directions at a time, however, this takes 1.6 times larger
than that just sending one
messages\cite{shida2012:mpi_kcomputer}. Thus, $5 / 1.6 \approx 3$ is 
observed. If a network interface hardware can only send one message at
a time, the communication performance degradation with using spare
node(s) can be relaxed. The network interface of BG/Q can also send
multiple messages at once\cite{Chen:2011:IBG:2063384.2063419}. Thus,
focusing on the maximum message collisions is considered to be
reasonable.

\comment{
There can be seen a step when the number of nodes exceed 400. This
phenomenon can be explained that the MPI for the K computer switches
internal protocol according to the number of hops
counts\cite{sumimoto-k}. According to the Sumimoto's presentation, the
threshold is 40. The farthest hop count of 20x20 network is 40 and
this matches the observation. XXXXXXXXXXX
}

Again, the drawback of 0D sliding is the possibility of long distance
between the fault node and spare node. As shown in Figure
\ref{fig:0d-K}, the communication latency was observed as three times
worse, almost independently from the number of nodes. This phenomenon can
differ when the message pattern is not the same with the 5P-stencil,
and considering the distance of replacement, the communication
performance may degrade more severely due to heavy message congestion. 

\subsubsection{1D Sliding}

As described in the previous subsection, in the 0D sliding method,
even if the closest spare node is chosen, 
the distance from the failed node cannot be small enough. The 1D sliding
method can avoid this situation. Figure \ref{fig:1d-sliding} shows how
the 1D sliding works. When the node 21 goes down, instead of substituting
the node with a spare node in the 0D sliding, the nodes of the column (or
row) which includes the failed node from the next of the failed node
to the end of the column are shifted to the side of spare nodes by one
(upper left of the figure). In this way, the hop count in the
5D-stencil communication pattern is increased by only one. This is much
smaller than the cases with the 0D sliding method. 

\begin{figure}[ht]
\begin{center}
\psfig{figure=Figs/1D-Sliding.eps,width=85mm}
  \caption{Example of 1D sliding} 
  \label{fig:1d-sliding}
\end{center}
\end{figure}

In terms of hop counts, 1D sliding method is superior to the
0D sliding method, however, the recoverable number of fault nodes is
limited in some cases. Let us think about the case where the second
failure happens in 2D(2,1) spare node allocation, node 16 following
the node 21 failure in Figure
\ref{fig:1d-sliding} for example, and the sliding direction is column
way now, and then the third failure happens on node 15, then there is
no space left for the 1D sliding (top row of Figure
\ref{fig:1d-sliding}). This situation can be avoided by 
sliding to the column direction on the second failure (middle row of
Figure \ref{fig:1d-sliding}). 

The number of nodes when the third failure cannot be handled in this
1D sliding method is the product of the number of sliding on both
sliding directions. Thus, evenly distributed sliding directions is not
a good idea. The sliding directions must be uneven as much as
possible (see sample program codelet in Figure \ref{fig:1d-prog}).
Even by doing so, there is a case where the 1D sliding method can
handle up to 3 failures (bottom row in Figure \ref{fig:1d-sliding}).

\begin{figure}[ht]
\small\tt
\begin{verbatim}
bool try_1D_slide( Node fnode ) {
    if ( try_1D_slide_X( fnode) ) {
        return true;
    }
    if ( try_1D_slide_Y( fnode ) ) {
        return true;
    }
    /* more dimension(s) may follow */
    return try_0D_slide( fnode );
}
\end{verbatim}
\caption{C program of 1D sliding} 
\label{fig:1d-prog}
\end{figure}

The relation between the maximum number of message collisions and the
number of failed nodes with the 2D(2,1) spare node allocation can be
expressed as follows. Note that there can be some cases where this
method cannot handle the situation when the number of node failures is
larger than 3. 

\[
C_{max} = 2 + F_n
\]

\subsubsection{1D+ Sliding}

The situation where fourth or more failure cannot be handled in the 1D
sliding method can be avoided by modifying the 1D sliding method. The
left of Figure \ref{fig:1dp-sliding} is the same as the bottom right of
Figure \ref{fig:1d-sliding}. When the fourth failure of node 15
happens, the nodes 9, 10 and 10 are slided to the right to make a room
at the upper node of node 15, and node 15 is moved by one upward
(right of Figure \ref{fig:1dp-sliding}). This method is named {\it 1D+
  sliding} method. 

\begin{figure}[ht]
\begin{center}
\psfig{figure=Figs/1D+Sliding.eps,width=80mm}
  \caption{1D+ sliding} 
  \label{fig:1dp-sliding}
\end{center}
\end{figure}

\subsubsection{2D Sliding, 3D sliding, ... $q$D sliding}

In the 2D sliding method, rows and columns of node space are shifted by
one (bottom of Figure \ref{fig:node-substitution}) to make entire row
or column which include the failed node vacant. With this the 2D sliding
method, only one node failure in the 2D(1,1) or two node failure in
2D(2,1) can be handled. If the network has a higher dimension Cartesian
topology, then the 3D sliding or more higher order sliding can take place
in the same way.

With the XY routing, messages pass through the vacant row or
column orthogonally with the row or column. All message routes are the
same with the ones before the failure. Thus, unlike the 0D and 1D
sliding methods, although hop counts is increased by one, the message
congestion can be avoided. Further, this nature is independent from
the communication patterns of applications. 

\subsection{Comparison of proposed methods}\label{sec:comparison}

Figure \ref{fig:coll-hist} shows the histogram graphs over the number
of possible message collisions of 5P-stencil (no periodic boundary)
on the 0D (upper graph) and 1D sliding (lower graph) methods with the
node allocation patterns of 10x10, 20x20, 40x40, 80x80 and 160x160
(mesh topology). These numbers are obtained by software simulation. Y
axes are normalized so that frequency can easily be compared ones with
the different number of nodes. 

\begin{figure}[ht]
\begin{center}
\psfig{figure=Figs/Histograms.eps,width=70mm}
  \caption{Histogram over \# collisions (5P-stencil)} 
  \label{fig:coll-hist}
\end{center}
\end{figure}

As easily be seen, the larger the number of nodes, the higher the
frequency of the higher message collisions. This is because the low
message collision cases happen when a failed node is at 1) close to
the spare node set, 2) at the side of the node plane, or 3) at the
boundary of 5P-stencil. Thus, when the network topology is torus
and/or 5P-stencil computation has periodic boundaries, then the worst 
case frequency will get higher. Anyway, when the number of nodes is
very large, the worst number of message collisions happens in most
cases. 

Figure \ref{fig:comparison} shows the number of possible message
collisions over the number of failed nodes on the 0D sliding method with
the 2D(1,1) and 2D(2,1) spare node allocation, the 1D sliding method with
the 2D(2,1) allocation, and the 2D sliding with the 2d(2,1) allocation. 

\begin{figure}[ht]
\begin{center}
\psfig{figure=Figs/N-Failures.eps,width=82mm}
  \caption{Comparison of 0D, 1D and 2D sliding (5P-stencil, worst cases)} 
  \label{fig:comparison}
\end{center}
\end{figure}

As already described in Section \ref{sec:0d-sliding}, the number of
possible message collisions with 0D sliding with the 2D(1,1)
allocation over the number of failed node depend on the network
topology, mesh or torus when the number of faults is greater than
6 (upper left one in the figure). The 1D sliding method with the 2D(2,1)
spare node allocation can handle 3 failures in worst cases, however,
it can handle up to the number of spare nodes minus one, since the
spare node at the corner of the 2D(2,1) allocation cannot be used. And
the 2D sliding with 2D(2,1) can handle up to two failures in any cases.

\subsubsection{Hybrid method}

The substitution methods described so far are independent with the
others and can be applied simultaneously. Figure \ref{fig:hybrid-sliding}
shows the example of this hybrid method. The first and second failures
are handled by using the 2D sliding method (left and middle of the
figure), and the third failure is handled by using the 1D sliding
method (right of the figure). In this way, the message collisions can
be avoided up to the two failures and the job can survive with having
more number of failures.

\begin{figure}[ht]
\begin{center}
\psfig{figure=Figs/Hybrid.eps,width=80mm}
  \caption{Example of hybrid sliding} 
  \label{fig:hybrid-sliding}
\end{center}
\end{figure}

If stencil program users are willing to have the load redistribution
shown in Figure \ref{fig:node-failure}, then the vacant nodes as the
result of the first node failure handling can be used as spare nodes
and the 0D, 1D, 1D+ and 2D sliding methods can be applied for more
failures. With the users' sacrifice of extra programing effort of load 
distribution, the node utilization problem described in Section
\ref{sec:spare-alloc} to have spare nodes in advance can be avoided. 

However, as shown in Figure \ref{fig:hybrid-sliding}, the node-rank
mapping is going to be disordered as the number of failed nodes gets
larger. So far, we have been discussed about the possible message
collisions on each sliding method, however, it is not proved that the
number of collisions with the mixed sliding methods, and there can be a
possibility of having hot spot(s) in the network. To avoid this
situation, when the disorder reaches at some level, rank numbers of
nodes are reordered so that the hot spot(s) will be
removed. Unfortunately, we are unable to find such an algorithm at
this point. 

\subsection{Collective communication}

Up to now, the P2P communication performance of 5P-stencil
communication was the concern. In this subsection, the performance of
collective communication is focused. The communication patterns of
collective and 5P-stencil communications are very different. Thus, the
collective communication performance degradation might be some clues of
the other P2P communication patterns.

On the K computer, the various collective communications are tuned so
that the best performance can be obtained on the Tofu network. Not
only the K computer, the tuning of collective protocols is very
important on the Cray's Gemini network
too\cite{Pena:2013:ATM:2488551.2488564}. Conversely speaking, it 
is very difficult to have optimized collective protocols for any
possible node failure in advance.

On the K computer, each MPI collective communication has some
conditions on the physical shape of MPI communicator to have the tuned
protocol. Some of the conditions come from the special protocol tuned
for the Tofu network and the others come from implementation
issues. When a substitution of a failed node happens, one or more
conditions cannot be met. Thus, the collective communication
performance can degrade much more than that of stencil communication,
because the tuned special protocol cannot be applied, and because of
the message collisions described in the 5P-stencil cases described
above.

\begin{figure}[ht]
\begin{center}
\psfig{figure=Figs/Barrier-K.eps,width=80mm}
  \caption{Barrier Performance (12 \time 12 nodes)} 
  \label{fig:barrier-k}
\end{center}
\end{figure}

\begin{figure}[ht]
\begin{center}
\psfig{figure=Figs/Allreduce-K.eps,width=83mm}
  \caption{Allreduce Performance (12 \time 12 nodes, 256KiB)} 
  \label{fig:allreduce-k}
\end{center}
\end{figure}

Figure \ref{fig:barrier-k} shows the barrier performance and Figure
\ref{fig:allreduce-k} shows the allreduce performance (message size is
256Kib) on the K computer with having one and two node(s) substituted
in the 0D, 1D and 2d sliding methods. The number of nodes allocated in
those figures are 12 \time 12 (144), including the spare node set
with the 2D(2,1) method. In this benchmark program, all possible
failures are simulated and the worst times are shown at the upper
horizontal bars, the best times are at shown at the lower horizontal
bars, and the average of all possible faults are shown at the middle
horizontal bars. The Y-axes are the relative time compared with the
cases without having any failed node. So, bigger is worse.

In the barrier collective performance graph, the time variances of 2
faults are bigger than the cases of having one fault. Especially, the
worst time when using 1D sliding is close to 2. In the allreduce
performance graph, the 2D sliding method affects the performance only
a little bit. Thus, it seems that the 2D sliding method does not break
the conditions for the allreduce collective operation.

\section{Related Work}\label{sec:related-work}

Ferreira {\it et al.} indicated that dual hardware redundancy which can
utilize 50% of the hardware resource can be more efficient than
traditional checkpoint and restart in Exa-scale
systems\cite{Ferreira:2011:EVP:2063384.2063443} in the future. The
redundant part can be thought as spare node set. The difference is
that the redundant nodes are {\it hotter}-standby than the hot standby
nodes waiting for the intermediate computational status. The spare
nodes can be substituted for the failed nodes and almost immediately
take over the computation of the failed node.

Domke {\it et al.} showed communication performance difference in the
presence of network failure (link or switch) over different network
topologies and routing
algorithms\cite{Domke:2014:FND:2683593.2683659}. They analyze the
communication performance degradation when network links or switches
go down by software simulation and using TSUBAME2.0. In the K
computer, the Tofu direct network has redundant routes to bypass
failed node. However, the job is aborted and resubmit by the operating
system when the job is running on the failed part. ULFM only
supports node failures and it does not assume to have any
network failure. It is a long way to get the goal where
any kind of failures, node and network, can be mitigated.

\section{Discussion}

\subsection{Node utilization in a multi-job environment}
\label{label:multi-job}

In most supercomputers are operated with a batch job system, and many
jobs are running simultaneously. The spare node percentage numbers
shown in Figure \ref{fig:sparenode-percentage} is the percentage of a
job, not a system. So, if a machine has one million nodes and 100 jobs
are running (assume that these jobs run with 10,000 nodes, for
simplicity) then the overhead of having spare nodes can exceed 10
\%. 

The possibility for a job to have failure is proportional to the
number of nodes assigned to the job. Thus, the number of spare nodes
must also be proportional to the number of nodes of a job. Therefore,
the number of spare nodes by using the same spare node allocation
method proposed in this paper can be too many when the number of nodes
of a job is small. Ideally, the curves shown in Figure
\ref{fig:sparenode-percentage} must be horizontal lines.

\begin{figure}[ht]
\begin{center}
\psfig{figure=Figs/R-SpareNodes-combo.eps,width=60mm}
  \caption{Combination of spare node allocation methods} 
  \label{fig:combo-percentage}
\end{center}
\end{figure}

Figure \ref{fig:combo-percentage} shows a countermeasure of
this. Large jobs should have the higher-order spare node allocation
method and smaller jobs should have lower order spare node allocation
method so that the spare node percentage gets close to a flat line. In
the example of Figure \ref{fig:combo-percentage}, the spare node
percentage is kept in the rage from 2\% to 5 \% by the combination of
3D(3,1), 3D(2,1) and 3D(1,1) methods.

\subsection{User-level vs. system-level substitution}

So far in this paper, spare nodes are allocated by a job itself. This
is in our mind to develop a framework using something like ULFM and to
substitute fault node with a spare node by the framework. By using this,
users will not be bothered to handle the failure situation. This way to
allocate and substitute spare nodes by user programs is now named {\it
  user-level substitution}. Contrastingly, job management layer
outside of user programs can do this also. This way is called {\it
  system-level substitution}.

With the user-level substitution, user program is invoked on each spare
node and waiting for the migration data of failed node in a
hot-standby way. In this way, the call of {\tt MPI\_Comm\_spawn}
function, this is not available
always\cite{Teranishi:2014:TLF:2642769.2642774}, can be avoided. On
the other hand, system-level substitution can reduce the percentage of
spare nodes because spare node set can be shared by several
jobs. For example, spare nodes can be allocated at the boundaries of
jobs and the spare nodes substitute with the failed nodes of the jobs
sitting on both side of the boundary. However, the hot standby spare
nodes which can be available with user-level 
substitution is impossible. If the spare node set is not an adjacent
position then it may result in uncontrollable message collisions
with another job and unexpected communication performance degradation. 

\subsection{Job resubmission vs. fault mitigation}

One may argue that the job can be aborted and then resubmit the job
from a checkpoint, instead of mitigating the fault. In this way, the
utilization problem having a spare node set and the communication
performance degradation described so far in this paper can be
avoided. The job resubmission, however, may incur the long turn around
time especially when the system is heavily loaded. And the user-level
fault mitigation techniques such as described in
\cite{Davies:2011:HPL:1995896.1995923} can not be utilized. There are
many aspect and parameters to decide which is better. This paper only
reveals one aspect, possibility communication performance
degradation. It is still an open question, which
is better, job resubmission or fault mitigation.

\section{Summary and Future Work}

In this paper, for jobs which are very difficult to mitigate node
failure, spare node allocation and substitution with failed 
nodes methods are proposed, analyzed and compared in terms of
communication performance after the substitution. The substitution
methods are 0D, 1D, 1D+, 2D, and higher sliding. In the 5P-stencil
communication, the higher the order of the sliding method, the less
message collisions and less number of failed nodes able to
recover. Thus, the combination of those methods considered to be the
best. Some collective communications with having spare nodes were also
investigated.

One thing we missed in this paper is the relation between the actual
communication performance degradation and the message length. We
observed small or no performance degradation when the message sizes
are small. This phenomenon an be explained that jitters when to send
small messages from nodes prevent message collisions. We are
investigating this to prove this jitter assumption.

To substitute a filed node with a spare node, spare node must be
restarted in somehow. To slide nodes in 1D or 2D, computation on these
nodes must be migrated. We are planning to develop a framework to hide
the complexity to allocate and utilize spare node set describing in
this paper. The technical issues and actual overhead of doing so will
be reported when it is ready. 

\section*{Acknowledgment}
\comment{
We thank Franck Cappello, at Argonne National Laboratory. He gave us
some useful comments.
}
This research is partially supported by the CREST project of
JST (Japan Science and Technology Agency).

\bibliographystyle{IEEEtran}
\bibliography{ref,hori}

\end{document}
