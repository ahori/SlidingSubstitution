\documentclass[]{letter}

\newcommand{\AH}[1]{%
  $\langle\langle$#1$\rangle\rangle$}%

\newcommand{\REVIEW}[2]{
\begin{description}
\item{COMMENT:}
#1%
\item{ANSWER:}
#2%
\end{description}
}

\begin{document}

We thank the reviewers for their insightful comments.  We fixed all
the writing issues. Below are clarifications for some of the questions
asked.

\begin{description}
\item{Reviewer 1}

\REVIEW{Some of the details in the simulation and evaluation sections are
missing. Was a specific simulator used and, if so, which one? Did the
evaluation use microbenchmarks, if so, which ones? }
{We developed the simulator and the benchmark programs by ourselves. }

There are a few English language errors in the text. 

The Related Work section seems to be short. 
The Shadow Replication work by Rami Melhem et al. and ULFM
could be added to this section. The impact of communication avoiding
algorithms, which stencil codes are often targeted for, could be
discussed. 

\item{Reviewer 2}

Page 7, Line 30:
– The authors say that the number of combinations of failed node is
the factorial of the number of failed nodes aka. NC(F) = F! , this
subsumes the ordering of node-failures, but the experimental setup
(simulation or evaluation) do not appear to be taking this into
consideration. 

Page 3, Line 12:
– The authors do not, benchmark the proposals made for handling the
node failure against the existing fault tolerant framework.
– The functional demerits of GVR, ULFM were not mentioned other than
not being user friendly 
– Have the authors considered Reinit that appeared in IJHPCA last year
that makes Link for: New ULFM API’s user friendly.

Page 7, Line 37:
– The authors need to specify the basis of the choices they they make
for the simulation setup. 
– Forexampletheychoose100x100,12x12x12and24x24x24 are there any
implicit basis for these choices like  simulation time.

Page 4, Line 52:
– The definition of 2D(2,1) is clear and intuitive, but its not so for 3D(2,1)
– How does the SNA differ for 3D(2,1) and 2D(2,1)
– What is the definition for 3D(2,1) spare node allocation?

Page 9, Line 13:
– Having a explicit/concrete expression for slowdown ratio would
benefit the reader 

Page 11, Line 55:
– What does it mean to have a 3D-sliding for a SNA == 2D(2,1)

Page 12, Line 9
– In the discussion section (quite late), the authors mention that a
gure 3, that appears very early in the paper the spare node % are for
                                % individual jobs and not for multi
                                % job environment, this would not have
                                % been misleading if mentioned early 

• If component failures are common events are we still relying on
redundancy (spare-nodes) to tolerate hardware faults, clear
description of the migration sequence would be helpful. 

• The paper has considerable number of Typos and repeated words, there
were more than one instance where  the figures and the corresponding
texts describing them were not matching. 

– Some of the markings in the Legends in the Figures were not clear,
the use of color in the legend would be  bene cial for the reader

– English and writing style needs review as The article would benefit
from a close editing. 

– We found it difficult to follow a few of the author’s argument due to
the many stylistic and grammatical errors and grammatical errors. 

– The ``s<Nq'' is a typo, this should be ``s<Nq''

– The authors quote “white boxes” to be present in the  gure 6, which
is not explicit.


\item{Reviewer 3}

There are several ways in which this paper could be
improved. The 5P stencil benchmark is not really described in any
detail, and all simulation and actual performance results are shown in
relative terms. 

The paper could be improved by providing some more
details of the benchmark, including important characteristics such as
message size and message rate. Baseline results that show how the
performance is impacted by suboptimal allocation patterns would be
interesting as well. 

Since this work is most relevant for very
large-scale jobs, a baseline analysis of actual performance
degradation for different rank mappings would help set bounds for what
the performance implications are. Details on the benchmark would help
improve the ability of the benchmark to simulate actual
applications. 

There are several working and grammatical errors that
detract from the readability of the paper. It needs a through editing
pass before publication. 

\end{description}

\end{document}
